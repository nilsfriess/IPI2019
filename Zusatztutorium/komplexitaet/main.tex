\documentclass[aspectratio=169,handout]{beamer}
\usetheme{Madrid}

\usepackage[utf8]{inputenc}   %Zeichencodierung Text
\usepackage[T1]{fontenc}        %Schriftsatz Dokument
\usepackage[ngerman]{babel}     %Wortdefinitionen
\usepackage{amsmath,amssymb,amsthm}    %Standard-Mathe-Paket
\usepackage{listings}
\usepackage{color}
\usepackage[light]{kpfonts}

\def\R{\mathbb{R}} 
\def\N{\mathbb{N}}
\def\Z{\mathbb{Z}}
\def\K{\mathbb{K}}
\def\C{\mathbb{C}}
\def\Rp{\mathbb{R}_{+}}
\def\F{\mathbb{F}}
\def\I{\mathbb{I}}

\def\iff{~\Leftrightarrow~}

\def\eps{\mathtt{eps}}
% function spaces
\def\CC{\mathscr{C}}
% space of polynomials
\def\P{{\mathcal{P}}}

% integrals, derivatives
\def\d{\mathrm{d}}
\def\dx{\,\mathrm{d}x}
\def\dt{\,\mathrm{d}t} 

\def\rd{\text{rd}}

\def\Df{\mathrm{D}f}

\def\O{\mathcal{O}}

\def\rq{\mathrm{R}}

% duality pairing
\def\<{\langle}
\def\>{\rangle}

% Swap the definition of \abs* and \norm*, so that \abs
% and \norm resizes the size of the brackets, and the 
% starred version does not.
\makeatletter
\newcommand*{\tp}{%
  {\mathpalette\@transpose{}}%
}
\newcommand*{\@transpose}[2]{%
  % #1: math style
  % #2: unused
  \raisebox{\depth}{$\m@th#1\intercal$}%
}
\makeatother

\makeatletter
\newcommand*{\herm}{%
  {\mathpalette\@hermitian{}}%
}
\newcommand*{\@hermitian}[2]{%
  % #1: math style
  % #2: unused
  \raisebox{\depth}{$\m@th#1{\mathrm{H}}$}%
}
\makeatother

% vertical equals 
\newcommand{\verteq}{\rotatebox{90}{$\,=$}}
% underset with vertical equals
\newcommand{\equalto}[2]{\underset{\scriptstyle\overset{\mkern4mu\verteq}{#2}}{#1}}

% for fractions with bigger elements or nested fractions
\newcommand{\ffrac}[2]{\ensuremath{\frac{\displaystyle #1}{\displaystyle #2}}}

\DeclareMathOperator{\rang}{rang}
\DeclareMathOperator{\cond}{cond}
\DeclareMathOperator{\diag}{diag}
\DeclareMathOperator{\im}{im}
\DeclareMathOperator{\spr}{spr}

\DeclareMathOperator*{\argmax}{arg\,max}
\DeclareMathOperator*{\argmin}{arg\,min}

\newcommand*{\mat}[1]{\bm{#1}}
\let\oldvec=\vec
\renewcommand{\vec}[1]{\mathbf{#1}}

%%%%%%%%%%%%%%%%%%%%%%%%%%%%%%%%%%%%%%%%%%%%%%%%
%%%%% Theorem environments %%%%%%%%%%%%%%%%%%%%%
%%% Local Variables:
%%% mode: latex
%%% TeX-master: "main"
%%% End:
 

\definecolor{mygreen}{rgb}{0,0.6,0}
\definecolor{mygray}{rgb}{0.5,0.5,0.5}
\definecolor{mymauve}{rgb}{0.58,0,0.82}

\definecolor{highlight}{rgb}{1,0.6,0.6}

\lstset{ %
  basicstyle=\footnotesize,
  language=C++,
  stepnumber=1,
  showstringspaces=false,
  breaklines=true,                 % automatic line breaking only at whitespace
  captionpos=b,                    % sets the caption-position to bottom
  commentstyle=\color{mygreen},    % comment style
  escapeinside={\%*}{*)},          % if you want to add LaTeX within your code
  keywordstyle=\color{blue},       % keyword style
  stringstyle=\color{mymauve},     % string literal style
  xleftmargin=2em,
  frame=single
}

\AtBeginSection[]{
  \begin{frame}
  \vfill
  \centering
  \begin{beamercolorbox}[sep=8pt,center,shadow=true,rounded=true]{title}
    \usebeamerfont{title}\insertsectionhead\par%
  \end{beamercolorbox}
  \vfill
  \end{frame}
}

\title{Komplexität}
\begin{document}
\begin{frame}[fragile]
  \maketitle
\end{frame}

\section{Definition der Komplexitätsklassen}
\begin{frame}[fragile]
  Sei $f,g : \N \rightarrow \R_{+}$ zwei Funktionen. Dann ist
  \pause
  \begin{itemize}
  \item $g(n) = \Omega(f(n))$, wenn es Konstanten $c, N > 0$ gibt, sodass
    \[
      g(n) \ge c\, f(n) \qquad \text{ für alle } n \ge N \,.
    \]
    \pause
  \item $g(n) = \mathcal{O}(f(n))$, wenn es Konstanten $c, N > 0$ gibt, sodass
    \[
      g(n) \le c\, f(n) \qquad \text{ für alle } n \ge N \,.
    \]
    \pause
  \item $g(n) = \Theta(f(n))$, wenn
    \[
      g(n) = \Omega(f(n)) \quad \text{und} \quad g(n) = \mathcal{O}(f(n)) \,.
    \]
  \end{itemize}
\end{frame}

\begin{frame}
  \textbf{Bemerkungen}
  \begin{itemize}
  \item Die Konstanten müssen von $n$ unabhängig sein.
  \item Die wichtigste Klasse ist $\mathcal{O}(\cdot)$.
  \item $g(n)$ entspricht später der Laufzeit eines Algorithmus in Abhängigkeit der Eingabegröße $n$.\\
    Sagen dann: ``Der Algorithmus wächst wie $f(n)$''
   \end{itemize}
\end{frame}

\begin{frame}{Beispiele}
  Sei $g(n) = 2n$.\\ Behauptung: $g(n) = \mathcal{O}(n)$.\\[2ex]
  \pause
  \begin{proof}
    Müssen zeigen: $\exists C, N$, sd. $2n \le Cn$ für alle $n \ge N$.\\[2ex]
    \pause
    Mit $C = 3$ und $N = 1$ klappt's (zum Beispiel).
  \end{proof}
  \pause
  Das funktioniert natürlich genau so mit $g(n) = 3n$, $g(n) = 4n$ etc. Also
  \[
    \text{Für} \quad g(n) = kn \quad \text{ gilt } \quad g(n) = \mathcal{O}(n)\,.
  \]
  In anderen Worten:
  \begin{center}
    Der Koeffizient vor dem $n$ spielt keine Rolle.
  \end{center}
\end{frame}

\begin{frame}{Beispiele}
  Sei $g(n) = n$.\\ Behauptung: $g(n) = \mathcal{O}(\frac{1}{3}n)$.\\[2ex]
  \pause
  \begin{proof}
    Müssen zeigen: $\exists C, N$, sd. $n \le \frac{C}{3}n$ für alle $n \ge N$.\\[2ex]
    \pause
    Mit $C = 4$ und $N = 1$ klappt's (zum Beispiel).
  \end{proof}
  \pause
  \textbf{Frage:}\quad Gilt auch
  \[
     g(n) = \mathcal{O}(r\,n)\,,\quad r \text{ beliebig?}
   \]
   \pause
   Haben also gezeigt:
   \[
     \mathcal{O}(n) \, = \, \mathcal{O}(r\,n)
   \]
   für $r \in \R$ beliebig.
 \end{frame}

 \begin{frame}{Beispiele}
  Sei $g(n) = n^2$.\\ Behauptung: $g(n) \neq \mathcal{O}(n)$.\\[2ex]
  \pause
  \begin{proof}
    Es müsste $c$ existieren, sd. $n^2 \le cn$, bzw.
    $n \le c$.
    \pause
    Da $c$ von $n$ unabhängig sein muss, kann das nicht funktionieren
    (für festes $c$ ist $n = c+1 > c$).
  \end{proof}
  \pause
  Allgemeiner gilt:
  \[
    n^k \neq \mathcal{O}(n^l) \quad \text{für} \quad k > l \,.
  \]
\end{frame}

\begin{frame}{Beispiele}
  Sei $g(n) = n^2$.\\ Behauptung: $g(n) = \mathcal{O}(n^3)$.\\[2ex]
  \pause
  \begin{proof}
    Mit $c = 1$, $N = 1$ klappt's.
  \end{proof}
  \pause
  Allgemeiner gilt:
  \[
    n^l = \mathcal{O}(n^k) \quad \text{für} \quad k > l \,.
  \]
\end{frame}

\begin{frame}{Beispiele}
  Sei $g(n) = n^3 + 2n^2 + n$.\\ Behauptung: $g(n) = \mathcal{O}(n^3)$.\\[2ex]
  \pause
  \begin{proof}
    Mit $c = 4$, $N = 1$ klappt's (nachrechnen).
  \end{proof}
  \pause
  Allgemein: Ist $g(n) = \sum_{i=0}^m a_i n^i$ ein Polynom, dann gilt
  \[
    g(n) = \mathcal{O}(n^m)\,,
  \]
  das heißt für Polynome ist nur der Grad entscheidend.
\end{frame}

\section{Einfache Komplexitätsanalysen}
\begin{frame}[fragile]
  Wollen die Komplexität des folgenden Algorithmus bestimmen.
  \begin{lstlisting}[numbers=left]
    int sum(int a[], int n) {
      int res = 0;      
      for (int i=0; i < n; i++) {
        res = res + a[i];
      }
      return res;
    }
  \end{lstlisting}
  \pause
  Müssen also zunächst herausfinden, wie oft die Schleife ausgeführt wird und dann, wie groß die Komplexität der Befehle innerhalb der Schleife ist.
  \pause
  \begin{itemize}
  \item Befehl in der Schleife wird $n$-mal ausgeführt.
  \pause
  \item Befehl in der Schleife (Zeile 4) hängt nicht von $n$ hab, d.h. Komplexität $\mathcal{O}(1)$.
  \end{itemize}
  \pause
  Insgesamt also $n$ mal $\mathcal{O}(1)$, also $\mathcal{O}(n)$.
\end{frame}

\begin{frame}[fragile]
  Wollen die Komplexität des folgenden Algorithmus bestimmen.
  \begin{lstlisting}[numbers=left]
    int sum(Matrix mat) {
      int n = mat.size();
      int res = 0;
      
      for (int i=0; i < n; i++)
        for (int j=0; j < n; j++)
          res += mat[i][j];
      
      return res;
    }
  \end{lstlisting}
  \pause
  \begin{itemize}
  \item Äußere Schleife läuft $n$ mal durch.
  \pause
  \item Innere Schleife läuft $n$ mal durch.
  \pause
  \item Befehl in der Schleife (Zeile 7) hat konstante Komplexität.
  \end{itemize}
  \pause
  Insgesamt also $\mathcal{O}(n^2)$.
\end{frame}

\begin{frame}[fragile]
  In welcher Komplexitätsklasse liegt der folgende Algorithmus?
  \begin{lstlisting}[numbers=left]
    int func(int n) {
      int x = 2;
      
      for (int i=0; i < n / 2; i++)
        x = x * 2;
      
      return res;
    }
  \end{lstlisting}
  \pause
  Der Algorithmus liegt in
  \[
    \mathcal{O}(\frac{n}{2}).
  \]
  \pause
  Haben vorhin gezeigt:
  $\mathcal{O}(n/2) = \mathcal{O}(n)$, d.h. der Algorithmus liegt in $\mathcal{O}(n)$.
\end{frame}

\section{Kompliziertere Komplexitätsanalysen}

\begin{frame}[fragile]
  Wollen die Komplexität des folgenden Algorithmus bestimmen.
  \begin{lstlisting}[numbers=left]
    void selectionsort(std::vector<double> &list) {
      int n = list.size();
      for (int i=0; i < n-1; i++) {
        int min = i;
        for (int j = i+1; j < n; j++)
          if (list[j] < list[min])
            min = j;
        std::swap(list[i], list[min]);
      }
    }
  \end{lstlisting}
  \pause
  \textbf{Behauptung.}\quad Selection-Sort liegt in $\mathcal{O}(n^2)$.
  \begin{proof}
    Tafel.
  \end{proof}
\end{frame}

\begin{frame}[fragile]
  Wollen die Komplexität des folgenden Algorithmus bestimmen.
  \begin{lstlisting}[numbers=left]
    template <class T>
    void gausian_elimination(Matrix<T> &mat) {
      for (int i=0; i < mat.rows() - 1; i++) {
        for (int k=i+1; k < mat.rows(); k++) {
          double t = mat[k][i] / mat[i][i];
          for (int j=0; j < mat.cols(); j++) {
            mat[k][j] = mat[k][j] - t * mat[i][j];
          }
        }
      }
    }
  \end{lstlisting}
  \pause
  \textbf{Behauptung.}\quad Gauß-Elimination liegt in $\mathcal{O}(n^3)$
  \begin{proof}
    Übung.
  \end{proof}
\end{frame}

\begin{frame}[fragile]
  Wollen die Komplexität des folgenden Algorithmus bestimmen.
  \begin{lstlisting}[numbers=left]
    ...
    int k = 0;
    for (int i=n/2; i <= n; i++) {
      for (int j=2; j <= n; j = j*2) {
        k = k + n / 2;
      }
    }
    ...
  \end{lstlisting}
  \pause
  \textbf{Behauptung.}\quad Der Algorithmus liegt in $\mathcal{O}(n \log n)$.
  \begin{proof}
    Übung.\\
    \begin{itemize}
    \item Überlege zuerst wie oft die innere Schleife durchläuft ($j$ wird jedes mal verdoppelt).
    \item Überlege dann, wie oft die äußere Schleife durchläuft.
    \end{itemize}
  \end{proof}
\end{frame}

\begin{frame}[fragile]
  Wie oft wird \emph{IPI} ausgegeben?
  \begin{lstlisting}[numbers=left]
    void fun(int n) 
    { 
      for (int i=1; i<=n; i++) 
        for (int j=1; j<=log(i); j++) 
          std::cout << "IPI"; 
    }
  \end{lstlisting}
  \pause
  \textbf{Übung.}\\[2ex]
  Hinweise:
  \begin{itemize}
  \item $\log(a_1) + \log(a_2) + \dotsc + \log(a_n) = log(a_1 \cdot a_2 \cdot \dotsc \cdot a_n)$
  \item $\mathcal{O}(\log n!) = \mathcal{O}(n \log n)$\hfill(Stirling Formel)
  \end{itemize}
\end{frame}





\end{document}