\documentclass[
fontsize = 11pt,
paper    = a4,
BCOR     = 5mm,
DIV      = 12,
numbers  = noenddot,
]{scrartcl}

\usepackage[utf8]{inputenc}
\usepackage[T1]{fontenc}

\usepackage[ngerman]{babel}

\usepackage{newpxtext}
\usepackage[euler-digits]{eulervm}
\usepackage{microtype}
\recalctypearea

\usepackage{setspace}
\onehalfspacing

\addtokomafont{disposition}{\rmfamily}

\usepackage{amsmath,amssymb}
\usepackage{listings}
\usepackage{color}
\usepackage{hyperref}

\usepackage{ulem}

\definecolor{mygreen}{rgb}{0,0.6,0}
\definecolor{mygray}{rgb}{0.5,0.5,0.5}
\definecolor{mymauve}{rgb}{0.58,0,0.82}

\lstset{ %
  basicstyle=\footnotesize, 
  language=C++,
  %numbers=left,
  stepnumber=1,
  showstringspaces=false,
  breaklines=true,                 % automatic line breaking only at whitespace
  captionpos=b,                    % sets the caption-position to bottom
  commentstyle=\color{mygreen},    % comment style
  escapeinside={\%*}{*)},          % if you want to add LaTeX within your code
  keywordstyle=\color{blue},       % keyword style
  stringstyle=\color{mymauve},     % string literal style
}

\title{Zeiger und Arrays in C$++$}
\author{IPI Tutorium WS 19/20}
\date{}

\begin{document}
\maketitle
Anders als in früheren IPI Vorlesungen \sout{habt ihr leider das
  Pech}/ \sout{müsst ihr unglücklicherweise} dürft ihr euch an der
Herausforderung erfreuen, euch intensiv mit Zeigern
auseinanderzusetzen. Diese Zusammenfassung soll euch das ganze Thema
an ein paar einfachen Beispielen näher bringen.

\subsection*{Was sind Zeiger?}
Überhaupt nicht zu verstehen, was es mit Zeigern (engl. Pointern) auf
sich hat, ist zu Beginn leider ganz normal. Zunächst sind Zeiger aber
eigentlich auch nur Variablen, wie ihr sie bereits kennt. Allerdings
haben diese Variablen nun einen speziellen Zweck. Ein Zeiger enthält
nämlich nicht einfach nur eine Zahl, sondern zusätzlich die
Information, dass es sich bei dieser Zahl um eine Adresse handelt. Mit
Adresse meine ich hier die Stelle, an der die Variable im
Arbeitsspeicher eures Computer steht. Ein Zeiger alleine ergibt also
wenig Sinn, er wird erst nützlich, wenn er die Adresse einer anderen
Variablen bzw. eines anderen Objektes enthält. Das ganze sieht
bspw. so aus:
\begin{lstlisting}
  int zahl = 5;
  int* zeiger = &zahl;
\end{lstlisting}
Die Variable \lstinline{zahl} ist eine übliche
Integer-Variable. \lstinline{zeiger} hingegen ist (wie der Name
vermuten lässt) ein Zeiger, der nun die Adresse, der Variablen
\lstinline{zahl} enthält. Man sagt in diesem Fall ``\lstinline{zeiger}
zeigt auf \lstinline{zahl}''. Dabei macht das \& nichts anderes, als
die Adresse der Variablen auszugeben. Deshalb nennt man \& in diesem
Zusammenhang den Adressoperator.

Wie geht man nun mit Zeigern um? Betrachten wir dazu das folgende
Beispiel:
\begin{lstlisting}
  std::cout << zahl << std::endl;
  std::cout << zeiger << std::endl;
\end{lstlisting}
Was ist die Ausgabe des Programms? Die erste Zeile ist klar: es wird
einfach 5 ausgegeben. Die Ausgabe der zweiten Zeile ist jedoch sowas
wie 0x7fff31f823d0. Auch wenn es erstmal nicht so aussieht, aber das
ist einfach nur eine Zahl (in sog. Hexadezimalschreibweise). Bei
dieser Zahl handelt es sich um die Arbeitsspeicheradresse, die in
\lstinline{zeiger} gespeichert ist, d.\,h.\ die Adresse, an der die
Variable \lstinline{zahl} steht. Mit dieser Adresse können wir
natürlich nichts anfangen. Um nun den Wert der Variablen Zahl
auszulesen, geht man folgendermaßen vor
\begin{lstlisting}
  std::cout << *zeiger << std::endl;
\end{lstlisting}
Die Ausgabe hier ist jetzt wie erwartet 5. Man beachte, dass der *
Operator hier jetzt eine andere Bedeutung hat, also oben!  Oben gehört
der Operator quasi zum Typ \lstinline{int} dazu und macht klar, dass
es sich bei der Variablen um einen Zeiger handelt.  Hier sorgt der
Operator allerdings dafür, dass nicht die Adresse der Variablen auf
die der Zeiger zeigt ausgegeben wird, sondern deren Inhalt. Aus diesem
Grund heißt der Operator in diesem Fall Dereferenzierungsoperator.

Zeiger können natürlich auch für Variablen anderer Typen definiert
werden
\begin{lstlisting}
  double dValue = 2.0;
  double* dPointer = &dValue;
\end{lstlisting}
und auch für Objekte, d.\,h. Instanzen eines \lstinline{struct} (und
später auch einer \lstinline{class})
\begin{lstlisting}
  struct Complex {
    double real;
    double imag;
  };

  Complex c;
  c.real = 2;
  
  Complex* cPointer = &c;

  std::cout << cPointer->real << std::endl;
  // Ausgabe: 2
\end{lstlisting}
In diesem Beispiel sehen wir auch, wie sich der Zugriff auf die Member
eines Struct ändert, wenn man zeiger benutzt (\lstinline{c.real = 2}
vs. \lstinline{cPointer->real}).

\subsection*{Wofür benutzt man Zeiger?}
Nachdem wir geklärt haben, was Zeiger überhaupt sind, betrachten wir
jetzt ein paar Beispiele, die zeigen (lol) wofür man Zeiger benutzt.

Oben haben wir bisher Pointer nur verwendet, um auf die referenzierte
Variable zuzugreifen und sie auszulesen. Man kann die Variablen aber
natürlich auch verändern. Dazu muss man den Pointer, wie auch schon
beim Lesen, zunächst dereferenzieren.
\begin{lstlisting}
  int zahl;
  int* zeiger = &zahl;

  *zeiger = 10;

  std::cout << zahl << std::endl;
  // Ausgabe: 10
\end{lstlisting}
Das wird vor allem interessant, wenn Zeiger an Funktionen übergeben
werden.  Betrachten wir zunächst ein Beispiel \emph{ohne} Zeiger.
\begin{lstlisting}
  struct GanzGanzGrosseMatrix {
    ...
  };

  bool ist_invertierbar(GanzGanzGrossematrix matrix) {
    if (...)
    return true;
    else
    return false; 
  }

  int main() {
    GanzGanzGrossematrix m(10000); // 10000 . Zeilen x 10000 Spalten

    if (ist_invertierbar(m))
    std::cout << "Matrix ist invertierbar";
  }  
\end{lstlisting}
Die \lstinline{struct} soll hier einen Datentyp für eine sehr große
Matrix darstellen. Hat die Matrix Größe 10000x10000 und speichert
\lstinline{float}s, dann bräuchte die Matrix $\sim$ 40GB Platz im
Arbeitsspeicher. Wird nun die Funktion \lstinline{ist_invertierbar}
aufgerufen, so müssen (je nach Implementierung der \lstinline{struct})
unter Umständen 40GB im Arbeitsspeicher hin- und herkopiert werden,
was offensichtlich viel zu ineffizient ist. Implementiert man die
Funktion allerdings so
\begin{lstlisting}
  bool ist_invertierbar(GanzGanzGrossematrix *matrix) {
    if (...)
    return true;
    else
    return false; 
  }
\end{lstlisting}
und ruft sie so auf
\begin{lstlisting}
  if (ist_invertierbar(&m))
\end{lstlisting}
dann muss lediglich die Adresse des Objekts \lstinline{matrix} kopiert
werden, also nur ein paar Bytes.

C++ bietet dazu allerdings eine schönere Alternative an, nämlich die
sog. Referenzen. Referenzen sind \emph{syntaktischer Zucker}
(engl. syntactic sugar) für solche Beispiele wie oben, d.\,h.  sie
ermöglichen es dem/der Programmierer*in mithilfe einer einfacheren
Schreibweise, das selbe Ziel zu erreichen. Das sieht dann so aus:
\begin{lstlisting}
  bool ist_invertierbar(GanzGanzGrossematrix &matrix) {
    if (...)
    return true;
    else
    return false; 
  }
  ...
  if (ist_invertierbar(m))
\end{lstlisting}
Der große Vorteil von Referenzen gegenüber Zeigern ist, dass man am
Aufruf der Funktion nichts ändern muss (im Vergleich zur Version ohne
Zeiger). Aber Achtung! Der \&-Operator hat hier eine andere Bedeutung
als oben. Hier wird damit verdeutlicht, dass das Argument eine
Referenz ist und nicht wie oben, die Adresse einer Variablen
zurückgegeben. Ein weiterer Vorteil, bzw. eher ein Nachteil von
Zeigern ist, dass man mit der Zeiger-Version auch versehentlich die
Funktion mit einem Zeiger aufrufen kann, der auf gar kein Objekt
zeigt:
\begin{lstlisting}
  GanzGanzGrossematrix* matrixZeiger;

  if (ist_invertierbar(matrixZeiger))
\end{lstlisting}
Versucht die Funktion \lstinline{ist_invertierbar} auf diesen Zeiger
zuzugreifen (indem sie bspw. versucht den Zeiger zu dereferenzieren),
dann bricht das Programm (sehr wahrscheinlich) mit einem
sog. Segmentation-Error ab. Was in einem nicht initialisierter Zeiger
für eine Adresse steht ist immer unterschiedlich, wenn man Glück hat
enthält der Zeiger einfach nur den Wert 0 (der sog. Nullzeiger). Das
kann bei Referenzen nicht passieren, da man keine Referenzen erstellen
kann, die keiner Variablen zugordnet sind.

\paragraph{Einfach Verkettete Listen.} Für die Übungsaufgaben sind
besonderns sog. einfach und doppelt verkettete Listen von
Bedeutung. Diese können mithilfe von Zeigern sehr effizient
implementiert werden.

Zunächst betrachten wir einfach verkettete Listen. Dazu definieren wir
als erstes eine \lstinline{struct}, die ein Element der Liste
repräsentiert. Hier wollen wir \lstinline{int}-Variablen in der Liste
speichern.
\begin{lstlisting}
  struct Element {
    int value;
    Element* next = 0;
  };
\end{lstlisting}
Ein Element besteht also einerseits aus dem Wert, der im Element steht
und andererseits aus einem Zeiger auf seinen Nachfolger. Wir setzen
hier den Zeiger auf das nächste Element zunächst auf 0, um später
prüfen zu können, ob ein Elemente einen Nachfolger hat oder
nicht. Eine einfach verkettete Liste ist dann eine Liste, die einfach
einen Zeiger auf das erste Element enthält. Da dieser Zeiger dann auf
das nächste Element zeigt, und dieser wiederum auf das nächste
etc. kann man so die Liste von vorne nach hinten durchlaufen.
\begin{lstlisting}
  struct List {
    Element* first = 0;
  }
\end{lstlisting}
Um nun bspw. alle Elemente einer einfach verketteten Liste auszugeben,
kann man wie folgt vorgehen:
\begin{lstlisting}
  void printList(List list) {
    Element *current = list.first;
    while(current != 0) {
      std::cout << current->value << " ";
      current = current->next;
    }
    std::cout << std::endl;
  }
\end{lstlisting}
Um die Funktion auszuprobieren müssen wir ein paar Elemente erstellen
und in die Liste schreiben (zum Beispiel ans Ende der
Liste). Idealerweise hätten wir gerne eine Funktion die das für uns
übernimmt. Diese Funktion bekommt einerseits die Liste (als Zeiger, da
wir ja wirklich die Liste selbst verändern wollen und nicht nur eine
übergebene Kopie) und andererseits den Wert, der eingefügt werden
soll.
\begin{lstlisting}
  void insert(List *list, int value) {
    Element *newElement = new Element;
    newElement->value = value;

    Element *current = list->first;
    while(current->next != 0) {
      current = current->next;
    }

    current->next = newElement;
  }
\end{lstlisting}
Wir erstellen also ein neues Element, weisen ihm den übergebenen Wert
zu und durchlaufen die Liste von vorne bis zum Ende der Liste und
Fügen das Element dort als Nachfolger des derzeit letzen Elements ein.

Diese Funktion hat zwei Probleme: zum einen funktioniert sie nicht,
wenn die Liste noch leer ist und zum anderen dauert das Einfügen umso
länger, je länger die Liste ist. Deshalb fügt man in einer einfach
verketteten Liste Element üblicherweise am Anfang ein. Die
Vorgehensweise ist dabei folgendermaßen (ich gebe euch hier kein Beispiel, das
könnt ihr mal selbst probieren): Prüfe, ob Liste leer
ist. Falls ja, dann setze einfach das erste Element der Liste auf das
neu erstellte Element. Falls nein, dann mache das derzeit erste
Element der Liste zum Nachfolger des neuen Elements und aktualisiere
das erste Element der Liste. (Frage: Welche Komplexität hat das
Einfügen am Ende der Liste und welche das Einfügen am Anfang?)

Außerdem sollte einem sofort ein weiteres Problem auffallen. In der
obigen Funktion erstellt man neue Element mit \lstinline{new}. Wie in
der Vorlesung besprochen, muss es zu jedem \lstinline{new} immer auch
ein \lstinline{delete} geben. Das heißt ihr müsst auf jeden Fall noch
eine Funktion schreiben, die die ganze Liste löscht (d.h. die Liste
durchläuft und jedes für jedes Element\paragraph{Einfach Verkettete Listen.} Für die Übungsaufgaben sind
besonderns sog. einfach und doppelt verkettete Listen von
Bedeutung. Diese können mithilfe von Zeigern sehr effizient
implementiert werden.

Zunächst betrachten wir einfach verkettete Listen. Dazu definieren wir
als erstes eine \lstinline{struct}, die ein Element der Liste
repräsentiert. Hier wollen wir \lstinline{int}-Variablen in der Liste
speichern.
\begin{lstlisting}
  struct Element {
    int value;
    Element* next = 0;
  };
\end{lstlisting}
Ein Element besteht also einerseits aus dem Wert, der im Element steht
und andererseits aus einem Zeiger auf seinen Nachfolger. Wir setzen
hier den Zeiger auf das nächste Element zunächst auf 0, um später
prüfen zu können, ob ein Elemente einen Nachfolger hat oder
nicht. Eine einfach verkettete Liste ist dann eine Liste, die einfach
einen Zeiger auf das erste Element enthält. Da dieser Zeiger dann auf
das nächste Element zeigt, und dieser wiederum auf das nächste
etc. kann man so die Liste von vorne nach hinten durchlaufen.
\begin{lstlisting}
  struct List {
    Element* first = 0;
  }
\end{lstlisting}
Um nun bspw. alle Elemente einer einfach verketteten Liste auszugeben,
kann man wie folgt vorgehen:
\begin{lstlisting}
  void printList(List list) {
    Element *current = list.first;
    while(current != 0) {
      std::cout << current->value << " ";
      current = current->next;
    }
    std::cout << std::endl;
  }
\end{lstlisting}
Um die Funktion auszuprobieren müssen wir ein paar Elemente erstellen
und in die Liste schreiben (zum Beispiel ans Ende der
Liste). Idealerweise hätten wir gerne eine Funktion die das für uns
übernimmt. Diese Funktion bekommt einerseits die Liste (als Zeiger, da
wir ja wirklich die Liste selbst verändern wollen und nicht nur eine
übergebene Kopie) und andererseits den Wert, der eingefügt werden
soll.
\begin{lstlisting}
  void insert(List *list, int value) {
    Element *newElement = new Element;
    newElement->value = value;

    Element *current = list->first;
    while(current->next != 0) {
      current = current->next;
    }

    current->next = newElement;
  }
\end{lstlisting}
Wir erstellen also ein neues Element, weisen ihm den übergebenen Wert
zu und durchlaufen die Liste von vorne bis zum Ende der Liste und
Fügen das Element dort als Nachfolger des derzeit letzen Elements ein.

Diese Funktion hat zwei Probleme: zum einen funktioniert sie nicht,
wenn die Liste noch leer ist und zum anderen dauert das Einfügen umso
länger, je länger die Liste ist. Deshalb fügt man in einer einfach
verketteten Liste Element üblicherweise am Anfang ein. Die
Vorgehensweise ist dabei folgendermaßen (ich gebe euch hier kein
Beispiel, das könnt ihr mal selbst probieren): Prüfe, ob Liste leer
ist. Falls ja, dann setze einfach das erste Element der Liste auf das
neu erstellte Element. Falls nein, dann mache das derzeit erste
Element der Liste zum Nachfolger des neuen Elements und aktualisiere
das erste Element der Liste. (Frage: Welche Komplexität hat das
Einfügen am Ende der Liste und welche das Einfügen am Anfang?)

Außerdem sollte einem sofort ein weiteres Problem auffallen. In der
obigen Funktion erstellt man neue Element mit \lstinline{new}. Wie in
der Vorlesung besprochen, muss es zu jedem \lstinline{new} immer auch
ein \lstinline{delete} geben. Das heißt ihr müsst auf jeden Fall noch
eine Funktion schreiben, die die ganze Liste löscht (d.h. die Liste
durchläuft und jedes für jedes Element \lstinline{delete element}
aufruft; dabei muss man nur aufpassen, dass man nicht versehentlich
ein Element löscht, bevor man seinen Nachfolger zwischengespeichert
hat, sonst kann man den nämlich nicht mehr löschen.)

Auch ist es sinnvoll, eine Funktion zur Verfügung zu haben, die nur
ein Element am Anfang der Liste löscht (und das gelöschte Element
zurückgibt). Eine Funktion dafür könnte ungefähr so aussehen:
\begin{lstlisting}
  Element* remove(List *list) {
    Element* rem = list->first;
    list->first = list->first->next;
    return rem;
  }
  ...
  Element *rem = remove(&list);
\end{lstlisting}
Auch hier muss man wieder ein paar Sonderfälle beachte (leere Liste).
Hat man diese Funktion geschrieben, dann ist es natürlich einfach,
eine Funktion zu schreiben, die die ganze Liste löscht (indem man
einfach so lange das erste Element entfernt, bis es kein erstes
Element mehr gibt).
 \lstinline{delete element}
aufruft; dabei muss man nur aufpassen, dass man nicht versehentlich
ein Element löscht, bevor man seinen Nachfolger zwischengespeichert
hat, sonst kann man den nämlich nicht mehr löschen.)

Auch ist es sinnvoll, eine Funktion zur Verfügung zu haben, die nur
ein Element am Anfang der Liste löscht (und das gelöschte Element
zurückgibt). Eine Funktion dafür könnte ungefähr so aussehen:
\begin{lstlisting}
  Element* remove(List *list) {
    Element* rem = list->first;
    list->first = list->first->next;
    return rem;
  }
  ...
  Element *rem = remove(&list);
\end{lstlisting}
Auch hier muss man wieder ein paar Sonderfälle beachte (leere Liste).
Hat man diese Funktion geschrieben, dann ist es natürlich einfach,
eine Funktion zu schreiben, die die ganze Liste löscht (indem man
einfach so lange das erste Element entfernt, bis es kein erstes
Element mehr gibt).

\paragraph{Doppelt verkettete Listen.} Doppelt verkettete Listen sind
Listen, bei denen die Elemente folgendermaßen aussehen
\begin{lstlisting}
  struct Element {
    int value;
    Element *prev;
    Element *next;
  };
\end{lstlisting}
Das heißt, Elemente zeigen jetzt nicht nur auf ihren Nachfolger,
sondern auch auf ihren Vorgänger.  Die Liste sieht dann so aus:
\begin{lstlisting}
  struct List {
    Element* first;
    Element* last;
  };
\end{lstlisting}
Man speichert nun also neben dem ersten Element der Liste auch das
Letzte.  Alles, was im Abschnitt vorher diskutiert wurde gilt hier
genau so, nur eben in die andere Richtung. Da man nun zu jedem Element
immer Vorgänger und Nachfolger angeben muss, werden die Einfüge- und
Löschfunktionen natürlich ein bisschen komplizierter und man muss
mehrere Sonderfälle beachten. Fangt also am besten mit einer einfach
verketteten Liste an und erweitert sie Stück für Stück zu einer
doppelt verketteten Liste. Habt ihr die Liste erstellt, könnt ihr sie
folgendermaßen (wie auf dem Übungszettel verlangt) mit Zufallszahlen
befüllen:
\begin{lstlisting}
  #include <cstdlib> // fuer rand()
  ...
  DList list;
  for (int i = 0; i < 50000; i++) {
    Element* element = new Element;
    element->value = rand();
    insert(&dlist, element);
  }
\end{lstlisting}
Ich hoffe diese Erklärungen helfen euch ein bisschen weiter.  Ich
werden die Datei in den nächsten Tagen noch erweitern, und was zum
Zusammenhang zu Arrays und Zeigern schreiben. Für den aktuellen Zettel
ist aber erstmal der Teil oben wichtig.

\subsection*{Arrays == Zeiger?}
todo
\end{document}
