\documentclass[
fontsize = 11pt,
paper    = a4,
BCOR     = 5mm,
DIV      = 12,
numbers  = noenddot,
]{scrartcl}

\usepackage[utf8]{inputenc}
\usepackage[T1]{fontenc}

\usepackage[ngerman]{babel}

\usepackage{newpxtext}
\usepackage[euler-digits]{eulervm}
\usepackage{microtype}
\recalctypearea

\usepackage{setspace}
\onehalfspacing

\addtokomafont{disposition}{\rmfamily}

\usepackage{amsmath,amssymb}
\usepackage{listings}
\usepackage{color}
\usepackage{hyperref}

\usepackage[framemethod=TikZ, nobreak=false]{mdframed}
\mdfdefinestyle{mms}{%
  roundcorner=2mm,%
%  innertopmargin=-1.2mm,%
  backgroundcolor=gray!20}
\def\MMS{\marginpar{%
    \vspace{-1.5em}%
    \begin{mdframed}[style=mms]%
~~???%
\end{mdframed}}}

\definecolor{mygreen}{rgb}{0,0.6,0}
\definecolor{mygray}{rgb}{0.5,0.5,0.5}
\definecolor{mymauve}{rgb}{0.58,0,0.82}

\lstset{ %
  basicstyle=\footnotesize, 
  language=C++,
  %numbers=left,
  stepnumber=1,
  showstringspaces=false,
  breaklines=true,                 % automatic line breaking only at whitespace
  captionpos=b,                    % sets the caption-position to bottom
  commentstyle=\color{mygreen},    % comment style
  escapeinside={\%*}{*)},          % if you want to add LaTeX within your code
  keywordstyle=\color{blue},       % keyword style
  stringstyle=\color{mymauve},     % string literal style
}

\title{(Template-)Klassen}
\author{IPI Tutorium WS 19/20}
\date{}

\begin{document}
\maketitle
Ziel: Klasse definieren, die ein dynamisches Array verwaltet, sodass
der/die Entwickler/in sich nicht um die dynamische Speicherverwaltung
(aka \lstinline{new} und \lstinline{delete}) kümmern muss (ähnliches
Beispiel auch in der Vorlesung).

Objekte der Klasse sollen verwendet werden können, wie ``normale''
Arrays, d.h. folgendes Programm soll kompilieren:
\begin{lstlisting}
  int main() {
    int size;
    std::cout << "Arraygroesse eingeben";
    std::cin >> size;
  
    Array a(size);

    a[0] = 10;
    a[1] = 20;

    std::cout << a[0]; // 10
    std::cout << a[1]; // 20
  }
\end{lstlisting}
(Frage: Wieso kann man in diesem Programm kein statisch erzeugtes
Array verwenden?\MMS) Eine erste Version der Klasse könnte so aussehen
(die Zugriffe über [] funktionieren natürlich noch nicht):
\begin{lstlisting}
  class Array {
    private:
    int size;
    int *data;

    public:
    Array(int t_size) {
      size = t_size;
      data = new int[size];
    }
  };
\end{lstlisting}
Der Speicher, der im Konstruktor mit \lstinline{new} dynamisch
allokiert wird, muss auch irgendwo wieder freigegeben werden.  Wir
definieren also eine Methode \lstinline{void free()}, die diese
Aufgabe übernimmt:
\begin{lstlisting}
  class Array {
    public:
    ...
    void free() {
      delete[] data;
    }
  };
\end{lstlisting}
Man beachte die eckigen Klammern nach dem delete (warum sind die
wichtig?). Jetzt hat man zwar eine Methode zur Verfügung, die den
Speicher freigibt, allerdings muss man natürlich auch daran denken,
die Methode auch aufzurufen, wenn das jeweilige Objekt nicht mehr
benötigt wird (und man muss aufpassen, die Methode nicht aufzurufen
und das Objekt danach weiter zu benutzen). Zum Glück gibt es in C++
eine spezielle Methode, die sich eignet solche ``Aufräumarbeiten'' zu
erledigen: den Destruktor. Anstelle der Methode \lstinline{free()}
schreiben wir also:
\begin{lstlisting}
  class Array {
    public:
    ...
    ~Array() {
      delete[] data;
    }
  };
\end{lstlisting}
Frage: Wann wird der Destruktor aufgerufen?\MMS

Bevor wir Arrayzugriffe mit eckigen Klammern implementieren, schreiben
wir zunächst zwei Methoden \lstinline{int get(int pos)} und
\lstinline{void set(int pos, int val)}, die ein Arrayelement
zurückgeben bzw. setzen:
\begin{lstlisting}
  class Array {
    public:
    ...
    void set(int pos, int val) {
      data[pos] val;
    }

    int get(int pos) {
      return data[pos];
    }
  };
\end{lstlisting}
Frage: Kann man die Methoden so verändern, dass ungültige
Arrayzugriffe verhindert werden (also Zugriffe der Form
\lstinline{a.set(10,2)}, wenn \lstinline{a} bspw. nur die Größe 5
hat)?\MMS

Nun ist folgendes möglich:
\begin{lstlisting}
  Array a(10);
  a.set(0, 20);
  a.get(0); // 20
\end{lstlisting}
Um nun Zugriffe wie im Beispiel am Anfang zu ermöglichen,
\emph{überladen} (was bedeutet das?\MMS) wir den
eckige-Klammern-Operator:
\begin{lstlisting}
  class Array {
    public:
    ...
    int operator[](int pos) {
      return data[pos];
    }
  };
\end{lstlisting}
Wollen wir mit diesem Operator nun sowohl die get, als auch die
set-Methode ersetzen, gibt es noch ein Problem. Zuweisungen der Form
\begin{lstlisting}
  a[0] = 10;
\end{lstlisting}
führt zu dem etwas kryptischen Compilefehler \texttt{lvalue required
  as left operand of assignment}, was nichts anderes bedeutet, dass
wir hier versuchen einen Wert zuzuweisen, wo kein Wert zugewisen
werden kann.  Das ist auch völlig klar: der Rückgabewert des
\lstinline{operator[]} ist \lstinline|int| und wir versuchen dann
einem \lstinline|int|-Wert (nicht einer \lstinline|int|-Variablen) den
Wert 10 zuzuweisen, was natürlich keinen Sinn ergibt. Dieses Problem
lässt sich leicht lösen: anstatt den \emph{Wert} des jeweiligen
Arrayelements zurückzugeben, geben wir einfach eine \emph{Referenz}
auf diesen Wert zurück. Dafür müssen wir lediglich den Rückgabetyp des
Operators ändern, den Rest erledigt der Compiler:
\begin{lstlisting}
  class Array {
    public:
    ...
    int& operator[](int pos) {
      return data[pos];
    }
  };
\end{lstlisting}
Das Beispielprogramm vom Anfang kompiliert nun und funktioniert wie
erwartet. Vollständig ist unsere Klasse jedoch noch nicht. Betracte
dazu folgendes Beispiel:
\begin{lstlisting}[numbers=left]
  int main() {
    Array a(10);
    a[0] = 100;

    Array b(a);
    b[0] = 20;

    std::cout << a[0] << ", ";
    std::cout << b[0] << std::endl;
  }
\end{lstlisting}
Übersetzt man dieses Programm und führt es aus, passieren zwei
(evtl. unerwartete Dinge):
\begin{itemize}
\item Die Ausgabe ist \texttt{20, 20}
\item Das Programm stürzt mit einem Fehler ab
\end{itemize}
Was passiert hier? Zunächst wird das Objekt \texttt{a} der Größe 10
erstellt und dessen erstes Element auf 100 gesetzt. In Zeile 5 wird
nun das Objekt \texttt{b} erstellt, allerdings nicht mit dem oben
definierten Konstruktor, sondern aus dem Objekt \texttt{a}, mithilfe
des implizit definierten (d.h. automatisch vom Compiler generierten)
Copy-Konstruktors. Dieser Konstruktor macht folgendes: er durchläuft
alle Eigenschaften des Objekts \texttt{a} (also hier die Variablen
\texttt{size} und \texttt{data}) und kopiert deren Werte in das Objekt
\texttt{b}. Da das Array \texttt{data} in der Klasse jedoch als Zeiger
gespeichert wird, werden nicht die einzelnen Arrayelemente kopiert,
sondern nur der Zeiger. Dabei entsteht folgendes Problem: Es gibt nun
zwei Objekte (\texttt{a} und \texttt{b}) die \textbf{auf das selbe
  Array zeigen}. Nun erklärt sich auch die Ausgabe. Da die beiden
Objekte nun auf das selbe Array zeigen, führt eine Änderung des Arrays
in \texttt{b} auch zu einer Änderung des Arrays in \texttt{b} (denn es
handelt sich ja um das selbe Array!). Dieses Verhalten ist natürlich
(in der Regel) unerwünscht, d.h. wir müssen den Copy-Konstruktor nun
selbst definieren und das Array ordentlich kopieren:
\begin{lstlisting}
  class Array {
    public:
    ...
    Array(Array &other) {
      size = other.size;
      data = new int[size];
      for (int i=0; i<size; i++)
      data[i] = other.data[i];
    }
  };  
\end{lstlisting}
Nun funktioniert das Programm wie erwartet.  Frage: Aus welchem Grund
ist das Programm oben abgestürtzt?\MMS

Das selbe Problem tritt auch auf, wenn wir anstatt ein Objekt zu
kopieren, ein Objekt einem anderen zuweisen (was genau ist der
Unterschied?\MMS):
\begin{lstlisting}
  Array a(10);
  a[0] = 10;
  Array b(20);
  b[0] = 20;
  
  b = a;
  std::cout << b[0] << std::endl; // 10
\end{lstlisting}
Das Problem löst man analog zu oben, jetzt muss man allerdings den
Zuweisungsoperator definieren: \lstinline{Array& operator=(Array &other)}.
Warum ist es sinnvoll, dass der Zuweisungsoperator den
Rückgabetyp \lstinline{Array&} hat?\MMS

Das oben beobachtete Problem tritt übrigens immer auf, wenn in einer
Klasse dynamisch (also mit \lstinline|new| speicher allokiert wird).
Die Regel, dass man in diesem Fall die drei Methoden Destruktor,
Copy-Konstruktor und Zuweisungsoperator selbst definieren muss, ist
als \emph{Rule of Three} bekannt. Nun haben wir eine vollständige
Array-Klasse, die natürlich noch beliebig erweitert werden kann.
\begin{lstlisting}
  class Array {
    private:
    int *data;
    int size;
    public:
    Array(int t_size) {
      size = t_size;
      data = new int[size];
    }
    
    Array(Array &other) {
      size = other.size;
      data = new int[size];
      for (int i=0; i<size; i++)
      data[i] = other.data[i];
    }
    
    Array& operator=(Array &other) {
      // Uebung :)
    }
    
    int& operator[](int pos) {
      return data[pos]; 
    }
    
    ~Array() {
      delete[] data;
    }
  };
\end{lstlisting}

Schön wäre es natürlich, wenn man nicht nur \lstinline|int|-Arrays,
sondern Arrays beliebiger Typen verwalten könnte. Dazu templatisieren
wir die Klasse, d.h. überall wo wir den Datentyp des Arrays (also
\lstinline{int} verwenden, schreiben wir einfach \texttt{T} und
deklarieren die Klasse als Templateklasse):
\begin{lstlisting}
  template <class T>
  class Array {
    private:
    T *data;
    int size;
    public:
    Array(int t_size) {
      size = t_size;
      data = new T[size];
    }
    
    Array(Array &other) {
      size = other.size;
      data = new T[size];
      for (int i=0; i<size; i++)
      data[i] = other.data[i];
    }
    
    Array& operator=(Array &other) {
      // Uebung :)
    }
    
    T& operator[](int pos) {
      return data[pos];
    }
    
    ~Array() {
      delete[] data;
    }
  };
\end{lstlisting}
Frage: Anstatt \lstinline|template <class T>| kann man auch
\lstinline|template <typename T>| schreiben. Was ist der
Unterschied?\MMS

Um diese Klasse verwenden zu können, muss nun beim erstellen des
Objekts der Templateparameter gesetzt wernde, d.h.
\begin{lstlisting}
  Array<int> a(10); // int-Array mit 10 Elementen

  Array<double> b(20); // double-Array mit 20 Elementen
\end{lstlisting}
Natürlich können auch komplexere Datentypen verwendet werden, wie
bspw. die Klasse Complex aus der Vorlesung:
\begin{lstlisting}
  Array<Complex> c(3); // Array mit 3 komplexen Zahlen
\end{lstlisting}
Hier ist nun folgendes wichtig: Verwendet man alle drei Beispiele im
selben Programm (also \lstinline|Array<int> a(10)|,
\lstinline|Array<double> a(10)| und \lstinline|Array<Complex> a(10)|)
dann werden \textbf{zur Compilezeit} drei unterschiedliche Klassen
erzeugt, für jeden Templatetype eine. Das heißt, zur Laufzeit verhält
sich das Programm so, als hätte man selbst für jeden einzelnen Typ
eine separate Klasse geschrieben.  Das bedeutet, trotz der
Abstrahierung der Klasse (anstatt \lstinline{int}-Array ein Array
beliebigen Typs) wird das Programm nicht langsamer. Allerdings
benötigt das Programm länger zum Compilieren (da die ganzen Klassen ja
vom Compiler erstellt werden müssen), was bei verschachtelten
Templateklassen oft zu unerträglich lange Compilezeiten führt.

Ein weiterer Nachteil ist, dass Fehlermeldungen, die durch Templateklassen
erzeugt werden in der Regel völlig unbrauchbar sind. Außerdem können
verschachtelte Templateklasse (also bspw. Templateklassen die als
Templateparameter selbst wieder Templateklassen verwenden) schnell
sehr unübersichtlich werden.
\end{document}
